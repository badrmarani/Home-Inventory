% Appendix Template

\chapter{Code Python} \label{annexe1}

Tous les programmes sont disponibles sur la plateforme github.

\section{Fichier \texttt{auth.py}}

\begin{lstlisting}[
    language=Python
    showspaces=false,
    basicstyle=\ttfamily,
    basicstyle=\tiny,
    numbers=left,
    numberstyle=\tiny,
]
import mysql.connector
import os
import time

from app.interface import home


connection = mysql.connector.connect(
    host = 'localhost',
    port = '3306',
    database = 'Home',
    user = 'root',
    password = 'toor'
)
cursor = connection.cursor()

def gain_access(Username=None, Password=None):

    Username = str(input("Entrez votre nom d'utilisateur :\t"))
    Password = str(input("Entrez votre mot de passe :\t"))

    if not len(Username or Password) < 1:
        if True:
            cursor.execute('SELECT user_name, user_password FROM User')
            data = cursor.fetchall()
            if (Username, Password) in data :
                print("Connexion réussie !")
                time.sleep(1)
                home(Username)
            else :
                print("Le mot de passe ou le nom d'utilisateur n'existe pas")
        else:
            print("Erreur de connexion au système")

    else:
        print("Veuillez essayer de vous connecter à nouveau.")
        os.system('cls'); gain_access()

def register(Username=None, Password1=None, Password2=None):
    Username = input("Entrez votre nom d'utilisateur :\t")
    Password1 = input("Entrez votre mot de passe :\t")
    Password2 = input("Confirmez votre mot de passe :\t")


    if not len(Password1) <= 6:
        if not Username == None:
            if len(Username) < 1:
                print("Veuillez fournir un nom d'utilisateur")
                time.sleep(1); os.system('cls');
                register()
            else:
                if Password1 == Password2:

                    cursor.execute("insert into user (user_name, user_password, stock_id) values (%s, %s, 2)", (Username, Password1))
                    connection.commit()

                    print("Utilisateur créé avec succès !")
                    print("Veuillez vous connecter pour continuer...")
                    time.sleep(1); os.system('cls');
                    gain_access()

                else:
                    print("Les mots de passe ne sont pas identiques")
                    time.sleep(1); os.system('cls');
                    register()
    else:
        print("Mot de passe trop court")
        time.sleep(1); os.system('cls');
        register()
\end{lstlisting}

\section{Fichier \texttt{home\_app.py}}


\begin{lstlisting}[
    language=Python
    showspaces=false,
    basicstyle=\ttfamily,
    basicstyle=\tiny,
    numbers=left,
    numberstyle=\tiny,
]
import mysql.connector
import time
import os

connection = mysql.connector.connect(
    host = 'localhost',
    port = '3306',
    database = 'Home',
    user = 'root',
    password = 'toor'
)
cursor = connection.cursor()

def print_products() :
    cursor.execute("""
        SELECT product_name, category_name
        FROM Product, Category
        Where Product.category_id = Category.category_id
    """)
    data = cursor.fetchall()

    print("Nombre des produits : ", len(data))
    print("Liste des produits :")
    i = 0
    for d in data :
        print("Produit N°{:<12}\t{:<35}\t\tCatégorie : {:<12}".format(i, d[0], d[1]), end='\n'); i += 1

    if input("Revenir à l'interface d'accueil...") != '' :
        os.system('cls');


def print_results() :
    cursor.execute("""
        SELECT product_name, control_results
        FROM Product, Control, Stock
        Where
        	Control.stock_id = Stock.stock_id AND
            Stock.product_id = Product.product_id
        GROUP by product_name
    """)
    data = cursor.fetchall()

    print("Nombre des produits : ", len(data))
    print("Liste des produits, ainsi que leurs résultats:")
    i = 0
    for d in data :
        print("Produit N°{:<12}\t{:<30}\t\tRésultats : {:<12}".format(i, d[0], d[1]), end='\n'); i += 1

    if input("Revenir à l'interface d'accueil...") != '' :
        os.system('cls');

def change_password(Username = None) :
    Password1 = input("Entrez votre mot de passe :\t")
    Password2 = input("Confirmez votre mot de passe :\t")

    if Password1 == Password2 :

        cursor.execute("""
            UPDATE User
            SET
                user_password = {}
            WHERE
                user_name = {};
        """.format(Username, Password1))
        connection.commit()

        print("Mot de passe changée avec succès !")

    if input("Revenir à l'interface d'accueil...") != '' :
        os.system('cls');

def delete_product() :
    name = input("Entrez le nom de produit :\t")

    cursor.execute("""
        DELETE FROM User
        WHERE product_name = {}
    """.format(name))
    connection.commit()

    print("Produit supprimée avec succès !")


def add_new_product() :
    name = input("Entrez le nom de produit :\t")
    taille = input("Entrez le libellé de sa taille :\t")
    temp_min = input("Entrez sa temperature minimal :\t")
    temp_max = input("Entrez sa temperature maximal :\t")
    desc = input("Entrez une description de produit :\t")

    cursor.execute("""
        insert into Product
            (product_name, product_temperature_max, product_temperature_min, product_description, shape_id, information_id, category_id) values
            ({}, {}, {}, {}, {}, {}, {});
    """.format(name, taille, temp_min, temp_max, desc, 1, 1, 1))
    connection.commit()


    print("Mot de passe ajoutée avec succès !")


    if input("Revenir à l'interface d'accueil...") != '' :
        os.system('cls');

\end{lstlisting}
